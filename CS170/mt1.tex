\documentclass[10pt,twocolumn]{article}
\usepackage{amsmath,amssymb,fullpage,fancyhdr,moreverb}
\setlength{\parindent}{0in}

\renewcommand{\headrulewidth}{0.0pt}

\begin{document}
\begin{center}\textsc{sorting algorithms}\end{center} 
Find $k^{th}$ smallest element:
\begin{verbatimtab}[1]
fxn select(A[0..n-1],k)
	if n=1, return A[0]
	pick some v=A[i]
	A< : entries < v, length n<
	A= : entries = v, length n=
	A> : entries > v, length n>
	if k < n<, select(A<,k)
	if n< <= k < n>, return v
	else select(A>,k-n<-n=)
\end{verbatimtab}


\begin{center}\textsc{parallel computing}\end{center} 
Essentially, you have O(work done * num of sequential levels + work done * num of parallelized levels). If you have $p\geq n$, then you do $\frac{f(n)}{p} $ work at each part at $\log p$ levels. If $p < n$, then you have $\frac{f(n)}{p} $ work at $\log p$ parallelized levels + $\frac{f(n)}{p} $ work done at $\log \frac{n}{p} $ sequential levels.
\begin{center}\textsc{recurrences}\end{center} 
Given $f(x),g(x)$: $f = O(g)$ if $\frac{f}{g} \leq $ constant (has upper bound). This means that $g = \Omega(f)$. $f=\Theta(g)$ means that $f=O(g)$ and $g=O(f)$ ($f$ and $g$ differ by constant factor). $n^a$ dominates $n^b$ if $a>b$. Any exponential dominates any polynomial $3^n > n^5$. Any polynomial dominates any log ($n > (\log n)^3, n^2 > n\log n$). $f_1=O(g_1)$ and $f_2=O(g_2) \rightarrow f_1+f_2=O(g_1+g_2)$ (same with multiplication). $\log(n) = O(n)$. if $\frac{f(n)}{g(n)} \rightarrow$ finite nonzero constant as $n\rightarrow\infty$, then $f=\Theta(g)$. If $\frac{f(n)}{g(n)} \rightarrow 0$ as $n\rightarrow\infty$, then $f=O(g)$ but not $\Omega(g)$. If you have confusing exponents, try taking the log of both sides.

\[T(n) = \left\{ \begin{array}{l l} \Theta(f(n)) & \quad \mbox{if $f(n)=\Omega(n^d) \quad d>\log_b a$} \\ \Theta(f(n)\log n) & \quad \mbox{if $f(n) = \Theta(n^d) \quad d=\log_b a$} \\ \Theta(n^{\log_ba}) & \quad \mbox{if $f(n)=O(n^d) \quad d<\log_ba$}  \end{array} \right. \]
Compare $f(n)$ to $n^{\log_ba}$. This lets you know which one of these cases you want to choose $d$ to fit.

\begin{center}\textsc{summations}\end{center} 
$\displaystyle\sum_{i=0}^n = \frac{n(n+1)}{2} $ $\displaystyle\sum_{i=0}^n i^2 = \frac{n(n+1)(2n+1)}{6} $ $ \displaystyle\sum_{k=0}^{n-1}ar^k=a \frac{1-r^n}{1-r}$	$\displaystyle\sum_{i=1}^{\log n}n = n\log n$ $\displaystyle\sum_{i=1}^n \frac{1}{2^i} =1-\frac{1}{2^n} $
\begin{center}\textsc{FFT}\end{center} 
When doing FFT, you evaluate your polynomial at the $n^{th}$ roots of unity up to the closest power of 2: $\{1,i,-1,-i\}$ for $F_4$. Do do the \emph{inverse} FFT, evaluate it at the complex conjugate of that same vector: $\{1,-i,-1,i\}$ for $F_4$. We do $y=Fx$ to change polynomial from coefficients $x$ to values $y$.
\begin{verbatimtab}[1]
fxn FFT(A,w):
in: coefficient representation of polynomial
		A(x) of degree <= n-1, w=nth root of unity
out: value representation A(w^0)..A(w^n-1)
if w=1, return A(1)
express A(x) in form A_e(x^2) + xA_0(x^2)
FFT(A_e,w^2) to evaluate A_e at even powers of w
FFT(A_o,w^2) to evaluate A_o at even powers of w
for j=0..n-1:
	compute A(w^j) = A_e(w^2j)+w^j*A_o(w^2j)
return A(w^0)...A(w^n-1)
\end{verbatimtab}
FFT example on $P(x)=2x^0+3x^1+4x^2+5x^3$. Rewrite as even and odd terms: $P(x)=(2x^0+4x^2)+x(3x^0+5x^2)$. To make a recursive call here, we need to have $P(x)$ with degree$<2$, because we're moving from $F_4$ to $F_2$. We divide the exponents of $P_{\mbox{even,odd}}$ by 2, then plug in $x^2$ to preserve the polynomial. $P(x)=P_{\mbox{even}}(x^2)+xP_{\mbox{odd}}(x^2)$. Evaluate at the $2^{th}$ roots of unity $1,-1$, then we can evaluate $P(1,i,-1,-i)$ by using the fact that each of those squared is either $1$ or $-1$.
\begin{center}\textsc{graphs}\end{center} 
To find source, run DFS and the vertex w/ largest post[v] is in a source SCC. To find sink, do the same process on same graph w/ edges reversed. The source in $G_R$ will be a sink in $G$.
\end{document}
